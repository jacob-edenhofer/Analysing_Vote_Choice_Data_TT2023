% Options for packages loaded elsewhere
\PassOptionsToPackage{unicode}{hyperref}
\PassOptionsToPackage{hyphens}{url}
\PassOptionsToPackage{dvipsnames,svgnames,x11names}{xcolor}
%
\documentclass[
]{article}
\usepackage{amsmath,amssymb}
\usepackage{lmodern}
\usepackage{iftex}
\ifPDFTeX
  \usepackage[T1]{fontenc}
  \usepackage[utf8]{inputenc}
  \usepackage{textcomp} % provide euro and other symbols
\else % if luatex or xetex
  \usepackage{unicode-math}
  \defaultfontfeatures{Scale=MatchLowercase}
  \defaultfontfeatures[\rmfamily]{Ligatures=TeX,Scale=1}
  \setmainfont[]{fouriernc}
\fi
% Use upquote if available, for straight quotes in verbatim environments
\IfFileExists{upquote.sty}{\usepackage{upquote}}{}
\IfFileExists{microtype.sty}{% use microtype if available
  \usepackage[]{microtype}
  \UseMicrotypeSet[protrusion]{basicmath} % disable protrusion for tt fonts
}{}
\makeatletter
\@ifundefined{KOMAClassName}{% if non-KOMA class
  \IfFileExists{parskip.sty}{%
    \usepackage{parskip}
  }{% else
    \setlength{\parindent}{0pt}
    \setlength{\parskip}{6pt plus 2pt minus 1pt}}
}{% if KOMA class
  \KOMAoptions{parskip=half}}
\makeatother
\usepackage{xcolor}
\usepackage[margin=1in]{geometry}
\usepackage{color}
\usepackage{fancyvrb}
\newcommand{\VerbBar}{|}
\newcommand{\VERB}{\Verb[commandchars=\\\{\}]}
\DefineVerbatimEnvironment{Highlighting}{Verbatim}{commandchars=\\\{\}}
% Add ',fontsize=\small' for more characters per line
\usepackage{framed}
\definecolor{shadecolor}{RGB}{248,248,248}
\newenvironment{Shaded}{\begin{snugshade}}{\end{snugshade}}
\newcommand{\AlertTok}[1]{\textcolor[rgb]{0.94,0.16,0.16}{#1}}
\newcommand{\AnnotationTok}[1]{\textcolor[rgb]{0.56,0.35,0.01}{\textbf{\textit{#1}}}}
\newcommand{\AttributeTok}[1]{\textcolor[rgb]{0.77,0.63,0.00}{#1}}
\newcommand{\BaseNTok}[1]{\textcolor[rgb]{0.00,0.00,0.81}{#1}}
\newcommand{\BuiltInTok}[1]{#1}
\newcommand{\CharTok}[1]{\textcolor[rgb]{0.31,0.60,0.02}{#1}}
\newcommand{\CommentTok}[1]{\textcolor[rgb]{0.56,0.35,0.01}{\textit{#1}}}
\newcommand{\CommentVarTok}[1]{\textcolor[rgb]{0.56,0.35,0.01}{\textbf{\textit{#1}}}}
\newcommand{\ConstantTok}[1]{\textcolor[rgb]{0.00,0.00,0.00}{#1}}
\newcommand{\ControlFlowTok}[1]{\textcolor[rgb]{0.13,0.29,0.53}{\textbf{#1}}}
\newcommand{\DataTypeTok}[1]{\textcolor[rgb]{0.13,0.29,0.53}{#1}}
\newcommand{\DecValTok}[1]{\textcolor[rgb]{0.00,0.00,0.81}{#1}}
\newcommand{\DocumentationTok}[1]{\textcolor[rgb]{0.56,0.35,0.01}{\textbf{\textit{#1}}}}
\newcommand{\ErrorTok}[1]{\textcolor[rgb]{0.64,0.00,0.00}{\textbf{#1}}}
\newcommand{\ExtensionTok}[1]{#1}
\newcommand{\FloatTok}[1]{\textcolor[rgb]{0.00,0.00,0.81}{#1}}
\newcommand{\FunctionTok}[1]{\textcolor[rgb]{0.00,0.00,0.00}{#1}}
\newcommand{\ImportTok}[1]{#1}
\newcommand{\InformationTok}[1]{\textcolor[rgb]{0.56,0.35,0.01}{\textbf{\textit{#1}}}}
\newcommand{\KeywordTok}[1]{\textcolor[rgb]{0.13,0.29,0.53}{\textbf{#1}}}
\newcommand{\NormalTok}[1]{#1}
\newcommand{\OperatorTok}[1]{\textcolor[rgb]{0.81,0.36,0.00}{\textbf{#1}}}
\newcommand{\OtherTok}[1]{\textcolor[rgb]{0.56,0.35,0.01}{#1}}
\newcommand{\PreprocessorTok}[1]{\textcolor[rgb]{0.56,0.35,0.01}{\textit{#1}}}
\newcommand{\RegionMarkerTok}[1]{#1}
\newcommand{\SpecialCharTok}[1]{\textcolor[rgb]{0.00,0.00,0.00}{#1}}
\newcommand{\SpecialStringTok}[1]{\textcolor[rgb]{0.31,0.60,0.02}{#1}}
\newcommand{\StringTok}[1]{\textcolor[rgb]{0.31,0.60,0.02}{#1}}
\newcommand{\VariableTok}[1]{\textcolor[rgb]{0.00,0.00,0.00}{#1}}
\newcommand{\VerbatimStringTok}[1]{\textcolor[rgb]{0.31,0.60,0.02}{#1}}
\newcommand{\WarningTok}[1]{\textcolor[rgb]{0.56,0.35,0.01}{\textbf{\textit{#1}}}}
\usepackage{graphicx}
\makeatletter
\def\maxwidth{\ifdim\Gin@nat@width>\linewidth\linewidth\else\Gin@nat@width\fi}
\def\maxheight{\ifdim\Gin@nat@height>\textheight\textheight\else\Gin@nat@height\fi}
\makeatother
% Scale images if necessary, so that they will not overflow the page
% margins by default, and it is still possible to overwrite the defaults
% using explicit options in \includegraphics[width, height, ...]{}
\setkeys{Gin}{width=\maxwidth,height=\maxheight,keepaspectratio}
% Set default figure placement to htbp
\makeatletter
\def\fps@figure{htbp}
\makeatother
\setlength{\emergencystretch}{3em} % prevent overfull lines
\providecommand{\tightlist}{%
  \setlength{\itemsep}{0pt}\setlength{\parskip}{0pt}}
\setcounter{secnumdepth}{-\maxdimen} % remove section numbering
\newlength{\cslhangindent}
\setlength{\cslhangindent}{1.5em}
\newlength{\csllabelwidth}
\setlength{\csllabelwidth}{3em}
\newlength{\cslentryspacingunit} % times entry-spacing
\setlength{\cslentryspacingunit}{\parskip}
\newenvironment{CSLReferences}[2] % #1 hanging-ident, #2 entry spacing
 {% don't indent paragraphs
  \setlength{\parindent}{0pt}
  % turn on hanging indent if param 1 is 1
  \ifodd #1
  \let\oldpar\par
  \def\par{\hangindent=\cslhangindent\oldpar}
  \fi
  % set entry spacing
  \setlength{\parskip}{#2\cslentryspacingunit}
 }%
 {}
\usepackage{calc}
\newcommand{\CSLBlock}[1]{#1\hfill\break}
\newcommand{\CSLLeftMargin}[1]{\parbox[t]{\csllabelwidth}{#1}}
\newcommand{\CSLRightInline}[1]{\parbox[t]{\linewidth - \csllabelwidth}{#1}\break}
\newcommand{\CSLIndent}[1]{\hspace{\cslhangindent}#1}
\newcommand{\indep}{\perp \!\!\! \perp}
\usepackage[T1]{fontenc}
\usepackage{fouriernc}
\usepackage{setspace}\onehalfspacing
\usepackage{amsfonts}
\usepackage{dcolumn}
\usepackage{pifont}
\usepackage{booktabs}
\usepackage{placeins}
\usepackage{amssymb}
\usepackage{booktabs}
\usepackage{longtable}
\usepackage{array}
\usepackage{multirow}
\usepackage{wrapfig}
\usepackage{float}
\usepackage{colortbl}
\usepackage{pdflscape}
\usepackage{tabu}
\usepackage{threeparttable}
\usepackage{threeparttablex}
\usepackage[normalem]{ulem}
\usepackage{makecell}
\usepackage{xcolor}
\ifLuaTeX
  \usepackage{selnolig}  % disable illegal ligatures
\fi
\IfFileExists{bookmark.sty}{\usepackage{bookmark}}{\usepackage{hyperref}}
\IfFileExists{xurl.sty}{\usepackage{xurl}}{} % add URL line breaks if available
\urlstyle{same} % disable monospaced font for URLs
\hypersetup{
  pdftitle={Analysing Vote Choice Data},
  pdfauthor={Jacob Edenhofer},
  colorlinks=true,
  linkcolor={cyan},
  filecolor={Maroon},
  citecolor={Blue},
  urlcolor={magenta},
  pdfcreator={LaTeX via pandoc}}

\title{Analysing Vote Choice Data}
\usepackage{etoolbox}
\makeatletter
\providecommand{\subtitle}[1]{% add subtitle to \maketitle
  \apptocmd{\@title}{\par {\large #1 \par}}{}{}
}
\makeatother
\subtitle{Assignment 3}
\author{Jacob Edenhofer\footnote{\href{mailto:jacob.edenhofer@some.ox.ac.uk}{\nolinkurl{jacob.edenhofer@some.ox.ac.uk}}}}
\date{14 May 2023}

\begin{document}
\maketitle

\hypertarget{preliminaries}{%
\section{Preliminaries}\label{preliminaries}}

Let us import the necessary packages and the data:

\begin{Shaded}
\begin{Highlighting}[]
\CommentTok{\# packages }
\FunctionTok{library}\NormalTok{(tidyverse)}
\FunctionTok{library}\NormalTok{(here)}
\FunctionTok{library}\NormalTok{(modelsummary)}
\FunctionTok{library}\NormalTok{(haven)}
\FunctionTok{library}\NormalTok{(ggpubr)}
\FunctionTok{library}\NormalTok{(knitr)}
\FunctionTok{library}\NormalTok{(kableExtra)}
\FunctionTok{library}\NormalTok{(ggeffects)}
\FunctionTok{library}\NormalTok{(fixest)}
\FunctionTok{library}\NormalTok{(lme4)}
\FunctionTok{library}\NormalTok{(margins)}
\FunctionTok{library}\NormalTok{(bife)}

\CommentTok{\# data}
\NormalTok{ess789 }\OtherTok{\textless{}{-}} \FunctionTok{read\_dta}\NormalTok{(}\FunctionTok{paste0}\NormalTok{(}\FunctionTok{here}\NormalTok{(), }\StringTok{"/Data/ESS789.dta"}\NormalTok{))}
\end{Highlighting}
\end{Shaded}

\hypertarget{exercise-1}{%
\section{Exercise 1}\label{exercise-1}}

\hypertarget{section}{%
\subsection{1.1}\label{section}}

\textcolor{brown}{Make sure that variable gndr is a dummy taking values 0/1, Rescale variables ipequopt and impfree so that higher values measure higher importance, Create variable year for each wave of the survey, Create a categorical variable cohort that measure in which decade the respondent was born. Make thevariable have only 4 levels, one for each quartile of the year of birth distribution}

To prepare the data in the desired way, I run:

\begin{Shaded}
\begin{Highlighting}[]
\NormalTok{ess789\_mod }\OtherTok{\textless{}{-}}\NormalTok{ ess789 }\SpecialCharTok{\%\textgreater{}\%}
  \CommentTok{\# 1 for males, 0 for females}
  \FunctionTok{mutate}\NormalTok{(}\AttributeTok{gndr\_dummy =} \FunctionTok{ifelse}\NormalTok{(gndr }\SpecialCharTok{==} \DecValTok{1}\NormalTok{, }\DecValTok{1}\NormalTok{, }\DecValTok{0}\NormalTok{),}
         \AttributeTok{gndr\_dummy =} \FunctionTok{factor}\NormalTok{(gndr\_dummy),}
         \AttributeTok{ipeqopt\_recoded =} \FunctionTok{recode}\NormalTok{(}\FunctionTok{as.numeric}\NormalTok{(ipeqopt),}
                                   \StringTok{"1"} \OtherTok{=} \DecValTok{6}\NormalTok{, }
                                   \StringTok{"2"} \OtherTok{=} \DecValTok{5}\NormalTok{,}
                                   \StringTok{"3"} \OtherTok{=} \DecValTok{4}\NormalTok{, }
                                   \StringTok{"4"} \OtherTok{=} \DecValTok{3}\NormalTok{, }
                                   \StringTok{"5"} \OtherTok{=} \DecValTok{2}\NormalTok{, }
                                   \StringTok{"6"} \OtherTok{=} \DecValTok{1}\NormalTok{),}
         \AttributeTok{impfree\_recoded =} \FunctionTok{recode}\NormalTok{(}\FunctionTok{as.numeric}\NormalTok{(impfree),}
                                   \StringTok{"1"} \OtherTok{=} \DecValTok{6}\NormalTok{, }
                                   \StringTok{"2"} \OtherTok{=} \DecValTok{5}\NormalTok{,}
                                   \StringTok{"3"} \OtherTok{=} \DecValTok{4}\NormalTok{, }
                                   \StringTok{"4"} \OtherTok{=} \DecValTok{3}\NormalTok{, }
                                   \StringTok{"5"} \OtherTok{=} \DecValTok{2}\NormalTok{, }
                                   \StringTok{"6"} \OtherTok{=} \DecValTok{1}\NormalTok{),}
         \CommentTok{\# from ess website}
         \AttributeTok{year =} \FunctionTok{case\_when}\NormalTok{(essround }\SpecialCharTok{==} \DecValTok{7} \SpecialCharTok{\textasciitilde{}} \DecValTok{2014}\NormalTok{, }
\NormalTok{                          essround }\SpecialCharTok{==} \DecValTok{8} \SpecialCharTok{\textasciitilde{}} \DecValTok{2016}\NormalTok{, }
                          \ConstantTok{TRUE} \SpecialCharTok{\textasciitilde{}} \DecValTok{2018}\NormalTok{),}
         \CommentTok{\# quartiles obtained by running quantile(ess789\_mod$agea, na.rm = T)}
         \AttributeTok{cohort =} \FunctionTok{case\_when}\NormalTok{((agea }\SpecialCharTok{\textgreater{}=}\DecValTok{14} \SpecialCharTok{\&}\NormalTok{ agea }\SpecialCharTok{\textless{}} \DecValTok{35}\NormalTok{) }\SpecialCharTok{\textasciitilde{}} \StringTok{"[14, 35)"}\NormalTok{,}
\NormalTok{                            (agea }\SpecialCharTok{\textgreater{}=} \DecValTok{35} \SpecialCharTok{\&}\NormalTok{ agea }\SpecialCharTok{\textless{}} \DecValTok{50}\NormalTok{) }\SpecialCharTok{\textasciitilde{}} \StringTok{"[35, 50)"}\NormalTok{, }
\NormalTok{                            (agea }\SpecialCharTok{\textgreater{}=} \DecValTok{50} \SpecialCharTok{\&}\NormalTok{ agea }\SpecialCharTok{\textless{}} \DecValTok{64}\NormalTok{) }\SpecialCharTok{\textasciitilde{}} \StringTok{"[50, 64)"}\NormalTok{,}
                            \ConstantTok{TRUE} \SpecialCharTok{\textasciitilde{}} \StringTok{"[64, 114)"}\NormalTok{),}
         \AttributeTok{mnrchy\_factor =} \FunctionTok{factor}\NormalTok{(mnrchy), }
         \AttributeTok{eummbr\_factor =} \FunctionTok{factor}\NormalTok{(eummbr))}
\end{Highlighting}
\end{Shaded}

\hypertarget{section-1}{%
\subsection{1.2}\label{section-1}}

\textcolor{brown}{Look at the variables in the dataset: which ones vary at the individual level? Which at the country level? And which at the country-year level?}

I summarise the levels of variation for the different variables in table
1:

\begin{Shaded}
\begin{Highlighting}[]
\CommentTok{\# dataframe }
\NormalTok{df\_var }\OtherTok{\textless{}{-}} \FunctionTok{tribble}\NormalTok{(}\SpecialCharTok{\textasciitilde{}}\StringTok{"Variable"}\NormalTok{, }\SpecialCharTok{\textasciitilde{}}\StringTok{"Description"}\NormalTok{, }
                  \StringTok{"env"}\NormalTok{, }\StringTok{"level of green attitudes in a given country in a given year"}\NormalTok{,}
                  \StringTok{"cons"}\NormalTok{, }\StringTok{"level of social conservativism in a given country in a given year"}\NormalTok{,}
                  \StringTok{"eummbr"}\NormalTok{, }\StringTok{"EU membership dummy"}\NormalTok{, }
                  \StringTok{"mnrchy"}\NormalTok{, }\StringTok{"Consitutional monarchy dummy"}\NormalTok{,}
                  \StringTok{"ipequopt"}\NormalTok{, }\StringTok{"whether respondent believes that it is important that people are treated equally and have"}\NormalTok{,}
                  \StringTok{"impfree"}\NormalTok{, }\StringTok{"whether the respondent believes that it is important to make own decisions and be free"}\NormalTok{,}
                  \StringTok{"uemp5yr"}\NormalTok{, }\StringTok{" periods of unemployment experienced by the respondent in the five previous years"}\NormalTok{,}
                  \StringTok{"gndr"}\NormalTok{, }\StringTok{"respondent\textquotesingle{}s gender"}\NormalTok{, }
                  \StringTok{"agea"}\NormalTok{, }\StringTok{"respondent\textquotesingle{}s age"}\NormalTok{)}

\CommentTok{\# table}
\NormalTok{df\_var }\SpecialCharTok{\%\textgreater{}\%}
  \FunctionTok{kbl}\NormalTok{(}\AttributeTok{booktabs =}\NormalTok{ T, }\AttributeTok{caption =} \StringTok{"Summary table of levels of variation"}\NormalTok{) }\SpecialCharTok{\%\textgreater{}\%}
  \FunctionTok{kable\_styling}\NormalTok{(}\AttributeTok{latex\_options =} \StringTok{"hold\_position"}\NormalTok{) }\SpecialCharTok{\%\textgreater{}\%}
  \FunctionTok{pack\_rows}\NormalTok{(}\StringTok{"varies at country{-}year level"}\NormalTok{, }\DecValTok{1}\NormalTok{, }\DecValTok{2}\NormalTok{) }\SpecialCharTok{\%\textgreater{}\%}
  \FunctionTok{pack\_rows}\NormalTok{(}\StringTok{"varies at country level"}\NormalTok{, }\DecValTok{3}\NormalTok{, }\DecValTok{4}\NormalTok{) }\SpecialCharTok{\%\textgreater{}\%}
  \FunctionTok{pack\_rows}\NormalTok{(}\StringTok{"varies at individual level"}\NormalTok{, }\DecValTok{5}\NormalTok{, }\DecValTok{9}\NormalTok{)}
\end{Highlighting}
\end{Shaded}

\begin{table}[!h]

\caption{\label{tab:level-variation-table}Summary table of levels of variation}
\centering
\begin{tabular}[t]{ll}
\toprule
Variable & Description\\
\midrule
\addlinespace[0.3em]
\multicolumn{2}{l}{\textbf{varies at country-year level}}\\
\hspace{1em}env & level of green attitudes in a given country in a given year\\
\hspace{1em}cons & level of social conservativism in a given country in a given year\\
\addlinespace[0.3em]
\multicolumn{2}{l}{\textbf{varies at country level}}\\
\hspace{1em}eummbr & EU membership dummy\\
\hspace{1em}mnrchy & Consitutional monarchy dummy\\
\addlinespace[0.3em]
\multicolumn{2}{l}{\textbf{varies at individual level}}\\
\hspace{1em}ipequopt & whether respondent believes that it is important that people are treated equally and have\\
\hspace{1em}impfree & whether the respondent believes that it is important to make own decisions and be free\\
\hspace{1em}uemp5yr & periods of unemployment experienced by the respondent in the five previous years\\
\hspace{1em}gndr & respondent's gender\\
\hspace{1em}agea & respondent's age\\
\bottomrule
\end{tabular}
\end{table}

\textcolor{brown}{What’s the mean value of variables capturing the importance of freedom and equality for respondents?, Do they differ between countries with a Constitutional Monarchy and those without? And between EU members and non-members? Report your results in a nice, tidy table.}

To compare the mean values of \texttt{impfree\_recoded} and
\texttt{ipeqopt\_recoded} between respondents living in constitutional
monarchies, as opposed to those who do not, I run:

\begin{Shaded}
\begin{Highlighting}[]
\NormalTok{ess789\_mod }\SpecialCharTok{\%\textgreater{}\%}
\NormalTok{ dplyr}\SpecialCharTok{::}\FunctionTok{select}\NormalTok{(ipeqopt\_recoded, impfree\_recoded, mnrchy\_factor) }\SpecialCharTok{\%\textgreater{}\%}
 \FunctionTok{datasummary\_balance}\NormalTok{(}\SpecialCharTok{\textasciitilde{}}\NormalTok{mnrchy\_factor, }\AttributeTok{fmt =} \DecValTok{3}\NormalTok{,}
                     \AttributeTok{dinm\_statistic =} \StringTok{"p.value"}\NormalTok{,}
                     \AttributeTok{title =} \StringTok{"Comparing mean values between constitutional monarchies and republics"}\NormalTok{,}
                     \AttributeTok{output =} \StringTok{"kableExtra"}\NormalTok{,}
                     \AttributeTok{data =}\NormalTok{ .) }\SpecialCharTok{\%\textgreater{}\%}
  \FunctionTok{kable\_styling}\NormalTok{(}\AttributeTok{latex\_options =} \StringTok{"hold\_position"}\NormalTok{)}
\end{Highlighting}
\end{Shaded}

\begin{table}[!h]

\caption{\label{tab:mean-equal-free-monarchy}Comparing mean values between constitutional monarchies and republics}
\centering
\begin{tabular}[t]{lrrrrrr}
\toprule
\multicolumn{1}{c}{ } & \multicolumn{2}{c}{0 (N=74502)} & \multicolumn{2}{c}{1 (N=25394)} & \multicolumn{2}{c}{ } \\
\cmidrule(l{3pt}r{3pt}){2-3} \cmidrule(l{3pt}r{3pt}){4-5}
  & Mean & Std. Dev. & Mean & Std. Dev. & Diff. in Means & p\\
\midrule
ipeqopt\_recoded & 4.825 & 1.086 & 5.035 & 0.942 & 0.211 & <0.001\\
impfree\_recoded & 4.821 & 1.106 & 4.856 & 1.060 & 0.035 & <0.001\\
\bottomrule
\end{tabular}
\end{table}

Table 2 shows that respondents in constitutional monarchies, on average,
accord greater importance to equality than their counterparts in
republics, with the difference being significant at the 1\% level. The
same holds for \texttt{impfree}, though difference in means is small.

To compare the mean values of \texttt{impfree\_recoded} and
\texttt{ipeqopt\_recoded} between respondents living in EU member
states, as opposed to those who do not, I run:

\begin{Shaded}
\begin{Highlighting}[]
\NormalTok{ess789\_mod }\SpecialCharTok{\%\textgreater{}\%}
\NormalTok{  dplyr}\SpecialCharTok{::}\FunctionTok{select}\NormalTok{(ipeqopt\_recoded, impfree\_recoded, eummbr\_factor) }\SpecialCharTok{\%\textgreater{}\%}
  \FunctionTok{datasummary\_balance}\NormalTok{(}\SpecialCharTok{\textasciitilde{}}\NormalTok{eummbr\_factor, }\AttributeTok{fmt =} \DecValTok{3}\NormalTok{,}
                      \AttributeTok{dinm\_statistic =} \StringTok{"p.value"}\NormalTok{, }
                      \AttributeTok{title =} \StringTok{"Comparing mean values between EU members and non{-}members"}\NormalTok{,}
                      \AttributeTok{output =} \StringTok{"kableExtra"}\NormalTok{, }
                      \AttributeTok{data =}\NormalTok{ .) }\SpecialCharTok{\%\textgreater{}\%}
  \FunctionTok{kable\_styling}\NormalTok{(}\AttributeTok{latex\_options =} \StringTok{"hold\_position"}\NormalTok{)}
\end{Highlighting}
\end{Shaded}

\begin{table}[!h]

\caption{\label{tab:mean-equal-free-eu}Comparing mean values between EU members and non-members}
\centering
\begin{tabular}[t]{lrrrrrr}
\toprule
\multicolumn{1}{c}{ } & \multicolumn{2}{c}{0 (N=8986)} & \multicolumn{2}{c}{1 (N=90910)} & \multicolumn{2}{c}{ } \\
\cmidrule(l{3pt}r{3pt}){2-3} \cmidrule(l{3pt}r{3pt}){4-5}
  & Mean & Std. Dev. & Mean & Std. Dev. & Diff. in Means & p\\
\midrule
ipeqopt\_recoded & 4.901 & 1.044 & 4.876 & 1.056 & -0.025 & 0.031\\
impfree\_recoded & 4.941 & 1.074 & 4.819 & 1.096 & -0.123 & <0.001\\
\bottomrule
\end{tabular}
\end{table}

Table 3 shows that respondents in EU member states, on average, accord
less importance to equality than their counterparts in non-EU member
states, with the difference being significant at the 1\% level. The same
holds for \texttt{impfree}, though difference in means is small.

\textcolor{brown}{Finally, for each observation, create a variable indicating how much more (or less) the respondent value freedom over equality}

To create this variable, I subtract \texttt{ipeqopt\_recoded} from
\texttt{impfree\_recoded}. This variable is zero for respondents who
agree to the same extent with both items, negative for those who agree
more strongly with \texttt{ipeqopt} than with \texttt{impfree}, and
positive for those who agree more strongly with \texttt{impfree} and
\texttt{ipeqopt}.

\begin{Shaded}
\begin{Highlighting}[]
\NormalTok{ess789\_mod }\OtherTok{\textless{}{-}}\NormalTok{ ess789\_mod }\SpecialCharTok{\%\textgreater{}\%}
  \FunctionTok{mutate}\NormalTok{(}\AttributeTok{free\_equal\_diff =}\NormalTok{ impfree\_recoded }\SpecialCharTok{{-}}\NormalTok{ ipeqopt\_recoded)}
\end{Highlighting}
\end{Shaded}

\hypertarget{section-2}{%
\subsection{1.3}\label{section-2}}

\textcolor{brown}{Which are the factors that better predict whether a respondent prefers freedom over equality? (Hint: build your dependent variable first). Plot the coefficients and comment their significance. Plot how the predicted probabilities of preferring freedom over equality change for male and female respondents conditionally on their experience of unemployment.}

My dependent variable, \texttt{free\_better\_dummy}, is a binary
variable that takes the value of one if \texttt{free\_equal\_diff} is
positive, i.e.~if a respondent agrees more strongly with
\texttt{impfree} than with \texttt{ipeqopt}, and is zero otherwise. I
then estimate four logit specifications, with that variable as my
dependent variable:

\begin{itemize}
\item
  I start by regressing \texttt{free\_better\_dummy} on respondents' age
  following the literature on long-term value changes (e.g.
  \protect\hyperlink{ref-inglehart2010changing}{Inglehart and Welzel
  2010}).
\item
  Then, I add dummy for respondents' gender, reflecting recent arguments
  that men and women have systematically different social attitudes
  (e.g. \protect\hyperlink{ref-anduiza2022sexism}{Anduiza and Rico
  2022}; \protect\hyperlink{ref-oshri2022risk}{Oshri et al. 2022}).
\item
  Next, I add a dummy for unemployment experience, given that adverse
  economic shocks may affect beliefs about equality and freedom. Since
  country's EU (non-)membership and its status as a constitutional
  monarchy might also influence respondents' social attitudes I include
  dummies for these as well. Finally, I include a country's overall
  level of social conservatism and environmental concern in a given year
  since these can be construed as proxies for the broader societal
  context within which individuals form their own attitudes.
\item
  The final model is almost identical to model three, except for
  \texttt{eummbr\_factor} and \texttt{mnrchy\_factor} being excluded.
  This is because model four includes country fixed effects, which
  control for all (un)observed factors that vary across countries, but
  are constant over time. Since \texttt{eummbr\_factor} and
  \texttt{mnrchy\_factor} are constant over time, their inclusion is
  rendered superfluous by the country fixed effects.
\end{itemize}

\begin{Shaded}
\begin{Highlighting}[]
\CommentTok{\# dependent variable}
\NormalTok{ess789\_mod }\OtherTok{\textless{}{-}}\NormalTok{ ess789\_mod }\SpecialCharTok{\%\textgreater{}\%}
  \FunctionTok{mutate}\NormalTok{(}\AttributeTok{free\_better\_dummy =} \FunctionTok{ifelse}\NormalTok{(free\_equal\_diff }\SpecialCharTok{\textgreater{}} \DecValTok{0}\NormalTok{, }\DecValTok{1}\NormalTok{, }\DecValTok{0}\NormalTok{),}
         \AttributeTok{uemp5yr\_factor =} \FunctionTok{factor}\NormalTok{(uemp5yr))}

\CommentTok{\# model }
\NormalTok{free\_better\_model1 }\OtherTok{\textless{}{-}} \FunctionTok{glm}\NormalTok{(free\_better\_dummy }\SpecialCharTok{\textasciitilde{}}\NormalTok{ agea,}
                         \AttributeTok{family =} \FunctionTok{binomial}\NormalTok{(}\AttributeTok{link =} \StringTok{"logit"}\NormalTok{),}
                         \AttributeTok{data =}\NormalTok{ ess789\_mod)}
\NormalTok{free\_better\_model2 }\OtherTok{\textless{}{-}} \FunctionTok{glm}\NormalTok{(free\_better\_dummy }\SpecialCharTok{\textasciitilde{}}\NormalTok{ agea }\SpecialCharTok{+}\NormalTok{ gndr\_dummy,}
                         \AttributeTok{family =} \FunctionTok{binomial}\NormalTok{(}\AttributeTok{link =} \StringTok{"logit"}\NormalTok{),}
                         \AttributeTok{data =}\NormalTok{ ess789\_mod)}
\NormalTok{free\_better\_model3 }\OtherTok{\textless{}{-}} \FunctionTok{glm}\NormalTok{(free\_better\_dummy }\SpecialCharTok{\textasciitilde{}}\NormalTok{ agea }\SpecialCharTok{+}\NormalTok{ gndr\_dummy }\SpecialCharTok{+}\NormalTok{ uemp5yr}
                          \SpecialCharTok{+}\NormalTok{ eummbr\_factor }\SpecialCharTok{+}\NormalTok{ mnrchy\_factor }\SpecialCharTok{+}\NormalTok{ cons }\SpecialCharTok{+}\NormalTok{ env,}
                         \AttributeTok{family =} \FunctionTok{binomial}\NormalTok{(}\AttributeTok{link =} \StringTok{"logit"}\NormalTok{),}
                         \AttributeTok{data =}\NormalTok{ ess789\_mod)}
\NormalTok{free\_better\_model4 }\OtherTok{\textless{}{-}} \FunctionTok{bife}\NormalTok{(free\_better\_dummy }\SpecialCharTok{\textasciitilde{}}\NormalTok{ agea }\SpecialCharTok{+}\NormalTok{ gndr\_dummy }\SpecialCharTok{+}\NormalTok{ uemp5yr }\SpecialCharTok{+}\NormalTok{ cons }\SpecialCharTok{+}\NormalTok{ env }\SpecialCharTok{|}\NormalTok{ cntry, }
                                 \AttributeTok{model =} \StringTok{"logit"}\NormalTok{, }\AttributeTok{data =}\NormalTok{ ess789\_mod)}
\CommentTok{\# coefficient plot }
\FunctionTok{modelplot}\NormalTok{(}\FunctionTok{list}\NormalTok{(free\_better\_model1, free\_better\_model2, }
\NormalTok{               free\_better\_model3, free\_better\_model4),}
               \AttributeTok{coef\_map =} \FunctionTok{c}\NormalTok{(}\StringTok{"agea"} \OtherTok{=} \StringTok{"Age of respondent"}\NormalTok{,}
                            \StringTok{"gndr\_dummy1"} \OtherTok{=} \StringTok{"Gender dummy"}\NormalTok{, }
                            \StringTok{"uemp5yr"} \OtherTok{=} \StringTok{"Unemployment ex{-}}\SpecialCharTok{\textbackslash{}n}\StringTok{perience in last five years"}\NormalTok{,}
                            \StringTok{"eummbr\_factor1"} \OtherTok{=} \StringTok{"EU dummy"}\NormalTok{, }
                            \StringTok{"mnrchy\_factor1"} \OtherTok{=} \StringTok{"Monarchy dummy"}\NormalTok{,}
                            \StringTok{"cons"} \OtherTok{=} \StringTok{"Social conservatism}\SpecialCharTok{\textbackslash{}n}\StringTok{in country{-}year"}\NormalTok{, }
                            \StringTok{"env"} \OtherTok{=} \StringTok{"Pro{-}environmental atti{-}}\SpecialCharTok{\textbackslash{}n}\StringTok{tudes in country{-}year"}\NormalTok{)) }\SpecialCharTok{+}
  \FunctionTok{geom\_vline}\NormalTok{(}\AttributeTok{xintercept =} \DecValTok{0}\NormalTok{, }\AttributeTok{linetype =} \StringTok{"dashed"}\NormalTok{) }\SpecialCharTok{+}
  \FunctionTok{expand\_limits}\NormalTok{(}\AttributeTok{x =} \SpecialCharTok{{-}}\FloatTok{0.5}\NormalTok{) }\SpecialCharTok{+}
  \FunctionTok{labs}\NormalTok{(}\AttributeTok{title =} \StringTok{"Correlates of valuing freedom more than equality"}\NormalTok{, }
       \AttributeTok{caption =} \StringTok{"Model 4 includes country fixed effects"}\NormalTok{) }\SpecialCharTok{+}
  \FunctionTok{theme}\NormalTok{(}\AttributeTok{legend.position =} \StringTok{"bottom"}\NormalTok{)}
\end{Highlighting}
\end{Shaded}

\includegraphics{AVCD-Assignment3-Edenhofer_files/figure-latex/free-better-correlates-1.pdf}

The coefficient plot implies four lessons:

\begin{itemize}
\item
  Gender is consistently a statistically significant (at the 5\% level)
  predictor of valuing freedom more than equality, with men being, on
  average, more likely to do so than females, holding all other included
  covariates constant. Substantively, the log odds are roughly 20\%
  (\(100*(exp(0.18)-1)\)) for men than for females.
\item
  Age and unemployment experience do not significantly predict
  preferring freedom over equality.
\item
  On average, respondents residing in EU countries are, compared to
  their non-EU counterparts, significantly less likely to prefer freedom
  over equality, holding all other included covariates constant.
  Similarly, respondents residing in constitutional monarchies are less
  likely to express such a preferences, relative to those living in
  republics. Substantively, the log odds are approximately 20\% lower
  (\(100*(exp(-0.239)-1)\)) for EU, as opposed to non-EU, respondents,
  and 30\% (\(100*(exp(-0.367)-1)\)) lower for respondents in
  constitutional monarchies.
\item
  The inclusion of country fixed effects leads to a strongly positive
  association between overall social conservatism in a given year and a
  preference for freedom over equality, with the log odds increasing by
  roughly 130\% for a unit increase in social conservatism
  (\(100*(exp(0.84)-1)\)). By contrast, the coefficient estimate for
  pro-environmental attitudes becomes insignificant when including
  country fixed effects, suggesting that the original positive
  association is driven by (un)observed confounders.
\end{itemize}

\textcolor{brown}{Plot how the predicted probabilities of preferring freedom over equality change for male and female respondents conditionally on their experience of unemployment.}

To plot the predicted probabilities, I use the \texttt{ggpredict()}
function applied to a simple regression of \texttt{free\_better\_dummy}
on the interaction between \texttt{gndr\_dummy1} and
\texttt{uemp5yr\_factor}.

\begin{Shaded}
\begin{Highlighting}[]
\CommentTok{\# data }
\NormalTok{ess789\_mod }\OtherTok{\textless{}{-}}\NormalTok{ ess789\_mod }\SpecialCharTok{\%\textgreater{}\%}
  \FunctionTok{mutate}\NormalTok{(}\AttributeTok{gndr\_dummy1 =} \FunctionTok{factor}\NormalTok{(gndr\_dummy, }\AttributeTok{levels =} \FunctionTok{c}\NormalTok{(}\StringTok{"0"}\NormalTok{, }\StringTok{"1"}\NormalTok{), }
                             \AttributeTok{labels =} \FunctionTok{c}\NormalTok{(}\StringTok{"Female"}\NormalTok{, }\StringTok{"Male"}\NormalTok{)))}
\CommentTok{\# model }
\NormalTok{free\_better\_model5 }\OtherTok{\textless{}{-}} \FunctionTok{glm}\NormalTok{(free\_better\_dummy }\SpecialCharTok{\textasciitilde{}}\NormalTok{ gndr\_dummy1}\SpecialCharTok{*}\NormalTok{uemp5yr\_factor,}
                         \AttributeTok{family =} \FunctionTok{binomial}\NormalTok{(}\AttributeTok{link =} \StringTok{"logit"}\NormalTok{),}
                         \AttributeTok{data =}\NormalTok{ ess789\_mod)}
\CommentTok{\# plot }
\FunctionTok{plot}\NormalTok{(}\FunctionTok{ggpredict}\NormalTok{(free\_better\_model5, }\AttributeTok{terms =} \FunctionTok{c}\NormalTok{(}\StringTok{"gndr\_dummy1"}\NormalTok{, }\StringTok{"uemp5yr\_factor"}\NormalTok{)),}
     \AttributeTok{connect.lines =}\NormalTok{ T) }\SpecialCharTok{+}
  \FunctionTok{scale\_colour\_discrete}\NormalTok{(}\StringTok{"Any period of unemployment or work seeking in the last five years?"}\NormalTok{,}
                        \AttributeTok{labels =} \FunctionTok{c}\NormalTok{(}\StringTok{"1"} \OtherTok{=} \StringTok{"Yes"}\NormalTok{,}
                                   \StringTok{"2"} \OtherTok{=} \StringTok{"No"}\NormalTok{)) }\SpecialCharTok{+}
  \FunctionTok{labs}\NormalTok{(}\AttributeTok{x =} \StringTok{"Gender"}\NormalTok{, }\AttributeTok{y =} \StringTok{"Predicted probability"}\NormalTok{, }
       \AttributeTok{title =} \StringTok{"Predicted probabilities for valuing freedom more than equality"}\NormalTok{) }\SpecialCharTok{+}
  \FunctionTok{expand\_limits}\NormalTok{(}\AttributeTok{y =} \FunctionTok{c}\NormalTok{(}\FloatTok{0.2}\NormalTok{, }\FloatTok{0.3}\NormalTok{)) }\SpecialCharTok{+}
  \FunctionTok{theme}\NormalTok{(}\AttributeTok{legend.position =} \StringTok{"bottom"}\NormalTok{)}
\end{Highlighting}
\end{Shaded}

\includegraphics{AVCD-Assignment3-Edenhofer_files/figure-latex/predicted-probabilities-plot-1.pdf}

The vertical differences between the point estimates represent the
marginal effect of unemployment experience for men and women
respectively. For women, the marginal effect of having experience
unemployment within the last five years is negative, while it is
positive for men. That is, men become more likely to prefer freedom over
equality after having experienced unemployment (the difference is
significant at the 5\% level, which can be seen by running
\texttt{summary(free\_better\_model5)}), with the reverse holding for
women.

\hypertarget{section-3}{%
\subsection{1.4}\label{section-3}}

\textcolor{brown}{Estimate the model above using year-level fixed effects: What do the year-level fixed effects exactly do?, What are the variables that change? How? And why those in particular?}

By including year fixed effects, we restrict our attention to
cross-country variation within each ESS wave. Doing so allows us to
account for (confounding) factors, both observable and unobservable,
that vary over time and are constant across countries, such as common
economic shocks. Hence, I run:

\begin{Shaded}
\begin{Highlighting}[]
\NormalTok{free\_better\_model6 }\OtherTok{\textless{}{-}} \FunctionTok{bife}\NormalTok{(free\_better\_dummy }\SpecialCharTok{\textasciitilde{}}\NormalTok{ agea }\SpecialCharTok{+}\NormalTok{ gndr\_dummy }\SpecialCharTok{+}\NormalTok{ uemp5yr }\SpecialCharTok{+}\NormalTok{ eummbr\_factor }\SpecialCharTok{+}\NormalTok{ mnrchy\_factor }\SpecialCharTok{+}\NormalTok{ cons }\SpecialCharTok{+}\NormalTok{ env }\SpecialCharTok{|}\NormalTok{ year, }
           \AttributeTok{model =} \StringTok{"logit"}\NormalTok{, }\AttributeTok{data =}\NormalTok{ ess789\_mod)}
\CommentTok{\# plot }
\FunctionTok{modelplot}\NormalTok{(}\FunctionTok{list}\NormalTok{(free\_better\_model2, free\_better\_model3, }
\NormalTok{               free\_better\_model4, free\_better\_model6),}
          \AttributeTok{coef\_map =} \FunctionTok{c}\NormalTok{(}\StringTok{"agea"} \OtherTok{=} \StringTok{"Age of respondent"}\NormalTok{,}
                       \StringTok{"gndr\_dummy1"} \OtherTok{=} \StringTok{"Gender dummy"}\NormalTok{, }
                       \StringTok{"uemp5yr"} \OtherTok{=} \StringTok{"Unemployment ex{-}}\SpecialCharTok{\textbackslash{}n}\StringTok{perience in last five years"}\NormalTok{,}
                       \StringTok{"eummbr\_factor1"} \OtherTok{=} \StringTok{"EU dummy"}\NormalTok{, }
                       \StringTok{"mnrchy\_factor1"} \OtherTok{=} \StringTok{"Monarchy dummy"}\NormalTok{,}
                       \StringTok{"cons"} \OtherTok{=} \StringTok{"Social conservatism}\SpecialCharTok{\textbackslash{}n}\StringTok{in country{-}year"}\NormalTok{, }
                       \StringTok{"env"} \OtherTok{=} \StringTok{"Pro{-}environmental atti{-}}\SpecialCharTok{\textbackslash{}n}\StringTok{tudes in country{-}year"}\NormalTok{)) }\SpecialCharTok{+}
  \FunctionTok{geom\_vline}\NormalTok{(}\AttributeTok{xintercept =} \DecValTok{0}\NormalTok{, }\AttributeTok{linetype =} \StringTok{"dashed"}\NormalTok{) }\SpecialCharTok{+}
  \FunctionTok{labs}\NormalTok{(}\AttributeTok{title =} \StringTok{"Correlates of valuing freedom more than equality"}\NormalTok{, }
       \AttributeTok{caption =} \StringTok{"Model 3 includes country fixed effects; model 4 includes year fixed effects."}\NormalTok{) }\SpecialCharTok{+}
  \FunctionTok{theme}\NormalTok{(}\AttributeTok{legend.position =} \StringTok{"bottom"}\NormalTok{)}
\end{Highlighting}
\end{Shaded}

\includegraphics{AVCD-Assignment3-Edenhofer_files/figure-latex/year-fe-free-better-1.pdf}

\textcolor{red}{change in variables}

\hypertarget{section-4}{%
\subsection{1.5}\label{section-4}}

\textcolor{brown}{If you were asked at which other level you would add fixed effects, what would you answer?}

Ideally, I would want to probe the robustness of the above results by
including year-wave fixed effects. In this way, we could control both
for country-specific, time-invariant (un)observable confounders (country
fixed effects), and for wave-specific, country-invariant (un)observable
confounders (wave fixed effects).

\hypertarget{exercise-2}{%
\section{Exercise 2}\label{exercise-2}}

\textcolor{brown}{Re-estimate the model above using year-level fixed effects. This time, however, use a different dependent variable: the level of country’s conservativism.}

I estimate four specifications with \texttt{cons} as the dependent
variable. I include fixed effects via the \texttt{feols()} function from
the \texttt{fixest} package, which is computationally more efficient
than the \texttt{plm()} function, and clusters the standard errors at
the level of the fixed effects (here the year level).

The justification of all additional covariates is, for the most part,
analogous to that offered in 1.3. The only exception is the inclusion of
\texttt{impfree\_recoded} and \texttt{ipeqopt\_recoded} in the final two
models. These variables are only contained in these models because they
may be strongly multi-collinear with other variables, thereby inflating
the standard errors of the coefficient estimates and increasing the risk
of type II errors. To mitigate this risk, I estimate specifications with
and without these two variables.

\begin{Shaded}
\begin{Highlighting}[]
\NormalTok{socio\_cons\_year\_fe1 }\OtherTok{\textless{}{-}} \FunctionTok{feols}\NormalTok{(cons }\SpecialCharTok{\textasciitilde{}}\NormalTok{ gndr\_dummy }\SpecialCharTok{+}\NormalTok{ agea }\SpecialCharTok{+}\NormalTok{ uemp5yr\_factor }\SpecialCharTok{|}\NormalTok{ year, }\AttributeTok{data =}\NormalTok{ ess789\_mod)}
\NormalTok{socio\_cons\_year\_fe2 }\OtherTok{\textless{}{-}} \FunctionTok{feols}\NormalTok{(cons }\SpecialCharTok{\textasciitilde{}}\NormalTok{ gndr\_dummy }\SpecialCharTok{+}\NormalTok{ agea }\SpecialCharTok{+}\NormalTok{ uemp5yr\_factor }\SpecialCharTok{+} 
\NormalTok{                              eummbr\_factor }\SpecialCharTok{+}\NormalTok{ mnrchy\_factor }\SpecialCharTok{|}\NormalTok{ year, }\AttributeTok{data =}\NormalTok{ ess789\_mod)}
\NormalTok{socio\_cons\_year\_fe3 }\OtherTok{\textless{}{-}} \FunctionTok{feols}\NormalTok{(cons }\SpecialCharTok{\textasciitilde{}}\NormalTok{ gndr\_dummy }\SpecialCharTok{+}\NormalTok{ agea }\SpecialCharTok{+}\NormalTok{ uemp5yr\_factor }\SpecialCharTok{+} 
\NormalTok{                              eummbr\_factor }\SpecialCharTok{+}\NormalTok{ mnrchy\_factor }\SpecialCharTok{+}\NormalTok{ impfree\_recoded }\SpecialCharTok{|}\NormalTok{ year, }\AttributeTok{data =}\NormalTok{ ess789\_mod)}
\NormalTok{socio\_cons\_year\_fe4 }\OtherTok{\textless{}{-}} \FunctionTok{feols}\NormalTok{(cons }\SpecialCharTok{\textasciitilde{}}\NormalTok{ gndr\_dummy }\SpecialCharTok{+}\NormalTok{ agea }\SpecialCharTok{+}\NormalTok{ uemp5yr\_factor }\SpecialCharTok{+} 
\NormalTok{                              eummbr\_factor }\SpecialCharTok{+}\NormalTok{ mnrchy\_factor }\SpecialCharTok{+}\NormalTok{ impfree\_recoded }\SpecialCharTok{+}\NormalTok{ ipeqopt\_recoded }\SpecialCharTok{|}\NormalTok{ year, }\AttributeTok{data =}\NormalTok{ ess789\_mod)}

\CommentTok{\# coefficient plot }
\FunctionTok{modelplot}\NormalTok{(}\FunctionTok{list}\NormalTok{(socio\_cons\_year\_fe1, socio\_cons\_year\_fe2, }
\NormalTok{                socio\_cons\_year\_fe3, socio\_cons\_year\_fe4),}
          \AttributeTok{coef\_map =} \FunctionTok{c}\NormalTok{(}\StringTok{"ipeqopt\_recoded"} \OtherTok{=} \StringTok{"Equal treatment/opportunity"}\NormalTok{, }
                       \StringTok{"impfree\_recoded"} \OtherTok{=} \StringTok{"Importance of freedom"}\NormalTok{, }
                       \StringTok{"mnrchy\_factor1"} \OtherTok{=} \StringTok{"Monarchy dummy"}\NormalTok{, }
                       \StringTok{"eummbr\_factor1"} \OtherTok{=} \StringTok{"EU dummy"}\NormalTok{, }
                       \StringTok{"uemp5yr\_factor2"} \OtherTok{=} \StringTok{"Unemployment dummy"}\NormalTok{, }
                       \StringTok{"agea"} \OtherTok{=} \StringTok{"Age"}\NormalTok{, }
                       \StringTok{"gndr\_dummy1"} \OtherTok{=} \StringTok{"Gender dummy"}\NormalTok{)) }\SpecialCharTok{+}
  \FunctionTok{geom\_vline}\NormalTok{(}\AttributeTok{xintercept =} \DecValTok{0}\NormalTok{, }\AttributeTok{linetype =} \StringTok{"dashed"}\NormalTok{) }\SpecialCharTok{+}
  \FunctionTok{labs}\NormalTok{(}\AttributeTok{title =} \StringTok{"Correlates of social conservatism at the country{-}year level"}\NormalTok{, }
       \AttributeTok{caption =} \StringTok{"All models include year fixed effects."}\NormalTok{) }\SpecialCharTok{+}
  \FunctionTok{theme}\NormalTok{(}\AttributeTok{legend.position =} \StringTok{"bottom"}\NormalTok{)}
\end{Highlighting}
\end{Shaded}

\includegraphics{AVCD-Assignment3-Edenhofer_files/figure-latex/socio-cons-year-fe-1.pdf}

The coefficient plot shows that only gender is a significant predictor
of social conservatism once year fixed effects are taken into account,
with males being slightly more likely than females to be socially
conservative.

\textcolor{brown}{Re-estimate the model above using country-level fixed effects. (Hint: what class is the variable for country? Is it the most appropriate?) Plot the coefficients: What does it change with respect with the model with year fixed effects? Why?}

The logic of the four models below is analogous to the previous
exercise, save for country fixed effects replacing year fixed effects.
As discussed above, country fixed effects net out all (un)observed,
country-specific factors that are constant over time. To illustrate
this, I have included \texttt{eummbr\_factor} and
\texttt{mnrchy\_factor}, which are constant over time within countries.
R automatically drops these variables since they are already accounted
for by means of the country fixed effects, which is why they are not
represented in the coefficient plot below.

\begin{Shaded}
\begin{Highlighting}[]
\NormalTok{socio\_cons\_cntry\_fe1 }\OtherTok{\textless{}{-}} \FunctionTok{feols}\NormalTok{(cons }\SpecialCharTok{\textasciitilde{}}\NormalTok{ gndr\_dummy }\SpecialCharTok{+}\NormalTok{ agea }\SpecialCharTok{+}\NormalTok{ uemp5yr\_factor }\SpecialCharTok{|}\NormalTok{ cntry, }\AttributeTok{data =}\NormalTok{ ess789\_mod)}
\NormalTok{socio\_cons\_cntry\_fe2 }\OtherTok{\textless{}{-}} \FunctionTok{feols}\NormalTok{(cons }\SpecialCharTok{\textasciitilde{}}\NormalTok{ gndr\_dummy }\SpecialCharTok{+}\NormalTok{ agea }\SpecialCharTok{+}\NormalTok{ uemp5yr\_factor }\SpecialCharTok{+} 
\NormalTok{                              eummbr\_factor }\SpecialCharTok{+}\NormalTok{ mnrchy\_factor }\SpecialCharTok{|}\NormalTok{ cntry, }\AttributeTok{data =}\NormalTok{ ess789\_mod)}
\NormalTok{socio\_cons\_cntry\_fe3 }\OtherTok{\textless{}{-}} \FunctionTok{feols}\NormalTok{(cons }\SpecialCharTok{\textasciitilde{}}\NormalTok{ gndr\_dummy }\SpecialCharTok{+}\NormalTok{ agea }\SpecialCharTok{+}\NormalTok{ uemp5yr\_factor }\SpecialCharTok{+} 
\NormalTok{                              eummbr\_factor }\SpecialCharTok{+}\NormalTok{ mnrchy\_factor }\SpecialCharTok{+}\NormalTok{ impfree\_recoded }\SpecialCharTok{|}\NormalTok{ cntry, }\AttributeTok{data =}\NormalTok{ ess789\_mod)}
\NormalTok{socio\_cons\_cntry\_fe4 }\OtherTok{\textless{}{-}} \FunctionTok{feols}\NormalTok{(cons }\SpecialCharTok{\textasciitilde{}}\NormalTok{ gndr\_dummy }\SpecialCharTok{+}\NormalTok{ agea }\SpecialCharTok{+}\NormalTok{ uemp5yr\_factor }\SpecialCharTok{+} 
\NormalTok{                              eummbr\_factor }\SpecialCharTok{+}\NormalTok{ mnrchy\_factor }\SpecialCharTok{+}\NormalTok{ impfree\_recoded }\SpecialCharTok{+}\NormalTok{ ipeqopt\_recoded }\SpecialCharTok{|}\NormalTok{ cntry, }\AttributeTok{data =}\NormalTok{ ess789\_mod)}

\CommentTok{\# models }
\FunctionTok{modelplot}\NormalTok{(}\FunctionTok{list}\NormalTok{(socio\_cons\_cntry\_fe1, socio\_cons\_cntry\_fe2, }
\NormalTok{               socio\_cons\_cntry\_fe3, socio\_cons\_cntry\_fe4),}
          \AttributeTok{coef\_map =} \FunctionTok{c}\NormalTok{(}\StringTok{"ipeqopt\_recoded"} \OtherTok{=} \StringTok{"Equal treatment/opportunity"}\NormalTok{, }
                       \StringTok{"impfree\_recoded"} \OtherTok{=} \StringTok{"Importance of freedom"}\NormalTok{, }
                       \StringTok{"mnrchy\_factor1"} \OtherTok{=} \StringTok{"Monarchy dummy"}\NormalTok{, }
                       \StringTok{"eummbr\_factor1"} \OtherTok{=} \StringTok{"EU dummy"}\NormalTok{, }
                       \StringTok{"uemp5yr\_factor2"} \OtherTok{=} \StringTok{"Unemployment dummy"}\NormalTok{, }
                       \StringTok{"agea"} \OtherTok{=} \StringTok{"Age"}\NormalTok{, }
                       \StringTok{"gndr\_dummy1"} \OtherTok{=} \StringTok{"Gender dummy"}\NormalTok{)) }\SpecialCharTok{+}
  \FunctionTok{geom\_vline}\NormalTok{(}\AttributeTok{xintercept =} \DecValTok{0}\NormalTok{, }\AttributeTok{linetype =} \StringTok{"dashed"}\NormalTok{) }\SpecialCharTok{+}
  \FunctionTok{expand\_limits}\NormalTok{(}\AttributeTok{x =} \SpecialCharTok{{-}}\FloatTok{0.005}\NormalTok{) }\SpecialCharTok{+}
  \FunctionTok{labs}\NormalTok{(}\AttributeTok{title =} \StringTok{"Correlates of social conservatism at the country{-}year level"}\NormalTok{, }
       \AttributeTok{caption =} \StringTok{"All models include country fixed effects."}\NormalTok{) }\SpecialCharTok{+}
  \FunctionTok{theme}\NormalTok{(}\AttributeTok{legend.position =} \StringTok{"bottom"}\NormalTok{)}
\end{Highlighting}
\end{Shaded}

\includegraphics{AVCD-Assignment3-Edenhofer_files/figure-latex/socio-cons-cntry-fe-1.pdf}

The coefficient plot demonstrates that, within a given country,
respondents' belief in equality is significantly and negatively
associated with social conservatism in that country in a given year,
while all other covariates are insignificant.

\hypertarget{section-5}{%
\subsection{2.3}\label{section-5}}

\textcolor{brown}{Random Effects: estimate the model using random effects for years and country}

To estimate the desired models, I estimate four specifications, where
the logic underpinning the choice of covariates is analogous to the
previous exercises. The only difference is that I use the
\texttt{lmer()} to include random effects for years and country.

\begin{Shaded}
\begin{Highlighting}[]
\NormalTok{socio\_cons\_re1 }\OtherTok{\textless{}{-}} \FunctionTok{lmer}\NormalTok{(cons }\SpecialCharTok{\textasciitilde{}}\NormalTok{ gndr\_dummy }\SpecialCharTok{+}\NormalTok{ agea }\SpecialCharTok{+}\NormalTok{ uemp5yr\_factor }\SpecialCharTok{+}\NormalTok{ (}\DecValTok{1} \SpecialCharTok{+}\NormalTok{ essround }\SpecialCharTok{|}\NormalTok{ cntry),}
                       \AttributeTok{data =}\NormalTok{ ess789\_mod,}
                       \AttributeTok{control =} \FunctionTok{lmerControl}\NormalTok{(}\AttributeTok{optimizer =} \StringTok{"nloptwrap"}\NormalTok{))}
\NormalTok{socio\_cons\_re2 }\OtherTok{\textless{}{-}} \FunctionTok{lmer}\NormalTok{(cons }\SpecialCharTok{\textasciitilde{}}\NormalTok{ gndr\_dummy }\SpecialCharTok{+}\NormalTok{ agea }\SpecialCharTok{+}\NormalTok{ uemp5yr\_factor }\SpecialCharTok{+}\NormalTok{ eummbr\_factor }\SpecialCharTok{+}\NormalTok{ mnrchy\_factor }\SpecialCharTok{+}
\NormalTok{                         (}\DecValTok{1} \SpecialCharTok{+}\NormalTok{ essround }\SpecialCharTok{|}\NormalTok{ cntry), }
                       \AttributeTok{control =} \FunctionTok{lmerControl}\NormalTok{(}\AttributeTok{optimizer =} \StringTok{"nloptwrap"}\NormalTok{),}
                       \AttributeTok{data =}\NormalTok{ ess789\_mod)}
\NormalTok{socio\_cons\_re3 }\OtherTok{\textless{}{-}} \FunctionTok{lmer}\NormalTok{(cons }\SpecialCharTok{\textasciitilde{}}\NormalTok{ gndr\_dummy }\SpecialCharTok{+}\NormalTok{ agea }\SpecialCharTok{+}\NormalTok{ uemp5yr\_factor }\SpecialCharTok{+}\NormalTok{ eummbr\_factor }\SpecialCharTok{+}\NormalTok{ mnrchy\_factor }\SpecialCharTok{+} 
\NormalTok{                        impfree\_recoded }\SpecialCharTok{+}\NormalTok{ ipeqopt\_recoded }\SpecialCharTok{+}\NormalTok{ (}\DecValTok{1} \SpecialCharTok{+}\NormalTok{ essround }\SpecialCharTok{|}\NormalTok{ cntry), }
                       \AttributeTok{control =} \FunctionTok{lmerControl}\NormalTok{(}\AttributeTok{optimizer =} \StringTok{"nloptwrap"}\NormalTok{),}
                       \AttributeTok{data =}\NormalTok{ ess789\_mod)}
\CommentTok{\# modelsummary }
\FunctionTok{modelplot}\NormalTok{(}\FunctionTok{list}\NormalTok{(socio\_cons\_re1, socio\_cons\_re2, socio\_cons\_re3),}
          \AttributeTok{coef\_map =} \FunctionTok{c}\NormalTok{(}\StringTok{"ipeqopt\_recoded"} \OtherTok{=} \StringTok{"Equal treatment/opportunity"}\NormalTok{, }
                       \StringTok{"impfree\_recoded"} \OtherTok{=} \StringTok{"Importance of freedom"}\NormalTok{, }
                       \StringTok{"mnrchy\_factor1"} \OtherTok{=} \StringTok{"Monarchy dummy"}\NormalTok{, }
                       \StringTok{"eummbr\_factor1"} \OtherTok{=} \StringTok{"EU dummy"}\NormalTok{, }
                       \StringTok{"uemp5yr\_factor2"} \OtherTok{=} \StringTok{"Unemployment dummy"}\NormalTok{, }
                       \StringTok{"agea"} \OtherTok{=} \StringTok{"Age"}\NormalTok{, }
                       \StringTok{"gndr\_dummy1"} \OtherTok{=} \StringTok{"Gender dummy"}\NormalTok{)) }\SpecialCharTok{+}
  \FunctionTok{geom\_vline}\NormalTok{(}\AttributeTok{xintercept =} \DecValTok{0}\NormalTok{, }\AttributeTok{linetype =} \StringTok{"dashed"}\NormalTok{) }\SpecialCharTok{+}
  \FunctionTok{expand\_limits}\NormalTok{(}\AttributeTok{x =} \FunctionTok{c}\NormalTok{(}\SpecialCharTok{{-}}\FloatTok{0.6}\NormalTok{, }\FloatTok{0.4}\NormalTok{)) }\SpecialCharTok{+}
  \FunctionTok{labs}\NormalTok{(}\AttributeTok{title =} \StringTok{"Correlates of social conservatism at the country{-}year level"}\NormalTok{, }
       \AttributeTok{caption =} \StringTok{"All models include random effects for countries and years."}\NormalTok{) }\SpecialCharTok{+}
  \FunctionTok{theme}\NormalTok{(}\AttributeTok{legend.position =} \StringTok{"bottom"}\NormalTok{)}
\end{Highlighting}
\end{Shaded}

\includegraphics{AVCD-Assignment3-Edenhofer_files/figure-latex/socio-cons-cntry-year-re-1.pdf}

\begin{itemize}
\tightlist
\item
  Interpretatio.
\end{itemize}

\hypertarget{some-theory}{%
\subsection{2.4 Some Theory}\label{some-theory}}

\textcolor{brown}{What do fixed effects account for? Specify: for years and countries/geographic regions, What do random effects account for?, Following Schimdt-Catran and Fairbrother, illustrate the structure of fixed effects.}

\FloatBarrier

\hypertarget{references}{%
\section*{References}\label{references}}
\addcontentsline{toc}{section}{References}

\hypertarget{refs}{}
\begin{CSLReferences}{1}{0}
\leavevmode\vadjust pre{\hypertarget{ref-anduiza2022sexism}{}}%
Anduiza, Eva, and Guillem Rico. 2022. {``Sexism and the Far-Right Vote:
The Individual Dynamics of Gender Backlash.''} \emph{American Journal of
Political Science}.

\leavevmode\vadjust pre{\hypertarget{ref-inglehart2010changing}{}}%
Inglehart, Ronald, and Christian Welzel. 2010. {``Changing Mass
Priorities: The Link Between Modernization and Democracy.''}
\emph{Perspectives on Politics} 8 (2): 551--67.

\leavevmode\vadjust pre{\hypertarget{ref-oshri2022risk}{}}%
Oshri, Odelia, Liran Harsgor, Reut Itzkovitch-Malka, and Or Tuttnauer.
2022. {``Risk Aversion and the Gender Gap in the Vote for Populist
Radical Right Parties.''} \emph{American Journal of Political Science}.

\end{CSLReferences}

\end{document}
